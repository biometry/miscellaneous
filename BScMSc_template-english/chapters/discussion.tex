\chapter{Discussion}
\label{ch:discussion}

Try to write introduction and discussion in a way that the reader doesn't really require methods and results. To do so, you have to use the first (few) sentences of the discussion to recapitulate your findings.

Discuss your results here, ideally structured into various sections, of which one is:


\section{English style}
Personally, I am used to British English and the following points relate to my view of how Britons use and write English. There are a few things that seem to be established (more or less) in scientific writing, but rarely adhered to by non-native writers.

\begin{itemize}
\item ``Data'' is plural (of singular ``datum''): ``Data \emph{were} collected in June.''
\item ``like'' is mostly a verb in science, not a preposition. In a comparative setting use ``such as'' instead: ``Stars, \emph{such as} our sun, shine brightly.''
\item The verb ``to fit'' has as past and participle ``fit'' and ``fit'', respectively. However, it is now common to write ``The model was fitted.'' There is some discussion, when to use ``fit'' and when ``fitted''. But ``fit'' is still correct, too.
\item Commata are used to make text more comprehensible. There shouldn't be too few or too many. Recently, a pandemic of putting a comma after ``i.e.'' or ``e.g.'' has taken hold. If you want a reader of the text to make a brief pause, then do use a comma, otherwise don't. Do \textbf{not} put a comma after every first word in a sentence! ``Tomorrow I'll read this book.'' - no comma required!
\item ``that'' is very rarely preceded by a comma: ``It was last night's rain that brought down most of the leaves.''
\item ``that'' does \emph{not} refer to people: ``The person \emph{who} handed me the coffee fully embraced the concept of politeness.''
\item For an awful lot of animals, male, female, young and groups have specific, dedicated terms (see, e.g., here: \url{https://www.adducation.info/mankind-nature-general-knowledge/collective-nouns-for-animals/}). Please use them. ``Baby'' only refers to the young of the apes (i.e. gorilla, chimpanzee, orangutan, gibbon and human). Similarly, try to avoid human terms for animals when technical terms exist: ``The gravid horse jumped over the lazy fox.''
\item Paragraphs should not run for more than half a page.
\item The position of an apostrophe is not arbitrary. And an apostrophe is not a back tick. ``Every single user's trust is important, as we need all users' compliance.'' 
\end{itemize}


\section{Conclusions}
\label{sec:conclusions}

  Your conclusions summarise your main findings as presented in the
  Discussion chapter. Put them here...

