\chapter{Methods}
\label{ch:methods}
You may want to start with a few, brief sentences of what you actually did, before going into the details over the next sections.

\section{Site}
\section{Species}
 Remember that Latin species names are set in italics, and only the genus name is capitalised: \emph{Lacerta viridis}.
\section{Statistical Analysis}

  Describe the methods and measurements that you used here.

\begin{itemize}
\item Units (m, kg, s) should be separated by a space from the value, except for \% and ° (or '' and ' for geographical coordinates): ``We weighed all bolders in the vicinity of the south pole ($-90$°N $0$°E) to an accuracy of 5 g, i.e. to an accuracy below 0.1\%.''

\item Write ``a 2 km $\times$ 2 km grid'', and not ``a 2 $\times$ 2 km grid'' (because the units need to be correct).

\item In the text, you can use \texttt{\textbackslash textsubscript} instead of the math mode (\$ ... \$) like so: CO\textsubscript{2} vs CO$_2$.

\item In equations, text should be displayed as text, not as mathematical symbols: $n_\text{total}$ or $df_\text{residuals}$, not $n_{total}$ or $df_{residuals}$.

\item Note that numbers may look different (are of different size!) in text and math mode, as does the minus: -2.0°C vs $-2.0$°C. Use the math mode consistently for numbers when these indicate an actual mathematical numeral, not an English word (such as 1776 or chapter 5, where you should use the text mode). For units and number representation you can use the package \textbf{siunits} for maximal standardisation: \num{2.34 x 5.67} or $\SI{-2}{\celsius}$. I regard this as overkill in most situation, though.
\end{itemize}  
 
 