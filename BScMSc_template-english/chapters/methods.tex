\chapter{Methods}
\label{ch:methods}
You may want to start with a few, brief sentences of what you actually did, before going into the details over the next sections.

\section{Site}
\section{Species}
 Remember that Latin species names are set in italics, and only the genus name is capitalised: \emph{Lacerta viridis}.
\section{Statistical Analysis}

  Describe the methods and measurements that you used here.

\begin{itemize}
\item Units (m, kg, s) should be separated by a space from the value, except for \% and ° (or '' and ' for geographical coordinates): ``We weighed all bolders in the vicinity of the south pole ($-90$°N $0$°E) to an accuracy of 5 g, i.e. to an accuracy below 0.1\%.'' There is a some variation to this, and my comment refers to how it is most commonly practiced in the UK. The Scientific Unit group advises for spaces for all but plane angles: \url{https://physics.nist.gov/cuu/Units/units.html}. This is a good page to briefly go through for all other types of unit and number styles.

\item Write ``a 2 km $\times$ 2 km grid'' or, if you must, ``a 2 $\times$ 2 km$^2$ grid'', and not ``a 2 $\times$ 2 km grid'' (because the units need to be correct; see also the last link to the SI units webpage).

\item In the text, you can use \texttt{\textbackslash textsubscript} instead of the math mode (\$ ... \$) like so: CO\textsubscript{2} vs CO$_2$. Same goes for \texttt{\textbackslash textdegree} and \texttt{\textbackslash textmu}.

\item In equations, text should be displayed as text, not as mathematical symbols: $n_\text{total}$ or $df_\text{residuals}$, not $n_{total}$ or $df_{residuals}$.

\item Note that numbers may look different (are of different size!) in text and math mode, as does the minus: -2.0°C vs $-2.0$°C. Use the math mode consistently for numbers when these indicate an actual mathematical numeral, not an English word (such as 1776 or chapter 5, where you should use the text mode). For units and number representation you can use the package \textbf{siunits} for maximal standardisation: \num{2.34 x 5.67} or $\SI{-2}{\celsius}$. I regard this as overkill in most situation, though.
\end{itemize}  
 
\section{Own contribution, and that of others}
Please make absolutely clear what \emph{you} have done, and what others have done! If, for example, samples were send to a different lab for analysis, then this should be stated in two different aspects: what was done in that other lab, and what kind of data you received back. The latter will explain how much post-processing had to be done on your side.

Author contributions are demanded in many scientific journals, in various styles and formats. The key point is to let the reader know what you did, and what you did not. You don't have to put this into a table, but it has to be communicated clearly. Typical elements of such author contributions may include:
\begin{enumerate}
	\item conceived the idea
	\item developed the formal theory
	\item designed the experiment
	\item recruited the subjects
	\item supervised the experiment
	\item carried out the experiment/simulations/measurements
	\item verified the methods
	\item collected the data
	\item analysed the data
	\item wrote the first draft
	\item contributed to writing the manuscript
\end{enumerate}
Non-relevant parts are ``secured the funding for the project'', ``supervised the PhD researcher'' and ``made some comments along the way'' or ``listened to a half-baked version presented in a seminar''. (If we are talking about a manuscript, such minor contributions should not lead to (``honorary'') co-authorship, according to funding guidelines of the DFG and many other funding bodies.)